\documentclass[12pt]{article}

% set margins and spacing
\addtolength{\textwidth}{1.3in}
\addtolength{\oddsidemargin}{-.65in} %left margin
\addtolength{\evensidemargin}{-.65in}
\setlength{\textheight}{9in}
\setlength{\topmargin}{-.5in}
\setlength{\headheight}{0.0in}
\setlength{\footskip}{.375in}
\renewcommand{\baselinestretch}{1.0}
\linespread{1.0}

% load miscellaneous packages
\usepackage{csquotes}
\usepackage[american]{babel}
\usepackage{booktabs}
\usepackage{threeparttable}
\usepackage[usenames,dvipsnames]{color}
\usepackage{graphicx,amsbsy,amssymb, amsmath, amsthm, MnSymbol,bbding,times, verbatim,bm,pifont,pdfsync,setspace,natbib}

% enable hyperlinks and table of contents
\usepackage[pdftex,
bookmarks=true,
bookmarksnumbered=false,
pdfview=fitH,
bookmarksopen=true,hyperfootnotes=false]{hyperref}

% define environments
\newtheorem{definition}{Definition}
\newtheorem{fact}{Fact}
\newtheorem{result}{Result}
\newtheorem{proposition}{Proposition}



\begin{document}
\title{Insert title here}
\author{Name 1\thanks{Syracuse University, Economics Department. Email: kbuzard@syr.edu.} \and Name 2\thanks{abc} \and Name 3\thanks{abc}}
\date{\vskip-.1in \today}
\maketitle

\vskip.3in
\begin{center} {\bf Abstract} \end{center}

\begin{quote}
{\small Insert abstract text here: 75-200 words, very high-level summary of your project. Look at the three most closely-related papers in your literature review to get an idea of what should be included.}
\end{quote}

\bigskip
\section{Introduction} \label{sec:introduction}

  Governments use both domestic taxes and tariffs to increase revenue.  As developed countries lower tariffs, an important question is how they replace the lost revenue from decreased tariffs.  Understanding this shift matters for trade policy and for how countries tax systems may be affected.  

This paper focuses on the relationship between domestic taxes on goods and services and tariff levels in developed countries.  The goal is to examine how domestic taxes relate to tariffs in developed countries and how this relationship depends on a country's GDP and it's import to GDP ratio.  

The analysis uses data from 11 countries: Switzerland, Korea, Australia, France, Norway, Israel, the United States, Belgium, Ireland, Canada, and New Zealand.  

Our hypothesis is that domestic taxes will increase in developed countries that reduce or limit tariffs and that this relationship will be stronger in those countries with high import to GDP ratios than in countries with low import to GDP ratios.  High-import countries rely more heavily on international trade for a large source of government revenue , so tariff reductions create larger revenue gaps that governments offset by increasing domestic consumption taxes.   Optimal taxation theory suggests that when tariff revenue falls, governments may raise domestic taxes in order to maintain overall revenue (Broadway, Maital and Prachowny, 2002).  This effect should be strongest in countries that rely heavily on tariff revenue to finance federal funding.

The results of our analysis show that the relationship between tariffs and domestic taxes is not statistically significant in in high import and low import developed countries.  The correlations between tariff levels and domestic tax revenues are weak.  Although high import countries show a slightly positive-correlation and low-import countries show a slightly negative one, neither pattern provides strong evidence that domestic taxes systemically rise when tariffs fall.  Overall, our data suggests that developed countries may not rely on domestic taxes to replace tariff revenue, or that any substitution effect is too small to detect in this sample.  

The remainder of the paper includes literature reviews, data, results, discussion, and conclusion.  The literature reviews analyze existing research on this topic.  The data section describes the data sources, variables, and how we analyzed the data.  The results section presents the descriptive statistics and tables.  In the discussion section, limitations of the analysis and the meaning of the data are explained.


\section{Literature Review} \label{sec:literature}

\setlength{\parindent}{20pt}The relationship between tariff revenue and domestic taxation is the main focus of this research. Examining whether developed countries substitute domestic taxes for tariff revenue, and whether this substitution pattern varies by import intensity is an important part in addressing the research question. Keen and Ligthart (2002) provide another key framework, arguing theoretically that public revenue can increase through trade liberalization if domestic consumption taxes are increased while consumer prices remain stable. However, they don't test whether high import countries behave differently. 

    There is a lot of research surrounding the revenue implications of trade and fiscal policy reform. Waglé (2011) determines that as countries move away from trade based taxes, they partially offset revenue losses with domestic sources of taxation, his sample includes low income countries for 25 years. our research differs by analyzing countries individually and explicitly  testing whether import strength affects fiscal substitution. 
    Thuy Tien Ho (2023) examines how tax revenue and economic growth interact with trade openness in developing countries, building on Keynesian demand theory to argue that reducing tariffs decreases government revenue and weakens aggregate demand. Their paper finds that trade openness has mixed effects on tax revenue depending on the stage of development and institutional capacity. This is important to our research for several reasons. first, their work demonstrates that the relationship between trade policy and fiscal outcomes varies significantly from country to country. Some countries experience significant revenue losses from tariff reductions while others do not, depending on their level of development and institutional strength. Second, while Ho assumes that developing countries increase domestic taxes to offset tariff losses. Our research tests whether fiscal substitution actually occurs in developed countries. our results show that even developed countries with much higher tax revenues do not systematically substitute domestic taxes for tariff revenue, suggesting that fiscal substitution is not an automatic response to trade liberalization even when countries can easily afford to do so. 

\setlength{\parindent}{20pt} Sova (2011) take a different angle by examining trade balance outcomes and whether Europe Agreements produce symmetric or asymmetric effects on exports and imports in Central and Eastern European countries. They found that association agreements have a positive and significant impact on exports and imports with the estimated coefficients being higher for imports than for exports, suggesting trade asymmetry. 


\section{Data}
\label{sec:data}
\begin{table}[htbp]\centering
\def\sym#1{\ifmmode^{#1}\else\(^{#1}\)\fi}
\caption{Summary Statistics}
\begin{tabular}{l*{5}{c}}
\hline\hline
                    &\multicolumn{1}{c}{(1)}&\multicolumn{1}{c}{(2)}&\multicolumn{1}{c}{(3)}&\multicolumn{1}{c}{(4)}&\multicolumn{1}{c}{(5)}\\
                    &\multicolumn{1}{c}{N}&\multicolumn{1}{c}{mean}&\multicolumn{1}{c}{sd}&\multicolumn{1}{c}{min}&\multicolumn{1}{c}{max}\\
\hline
year                &         220&        2010&       5.779&        2001&        2020\\
Exports (\% GDP)    &         220&       43.49&       26.23&       9.036&       133.3\\
Imports (\% GDP)    &         220&       40.08&       22.11&       13.15&       124.4\\
Domestic Tax Revenue (\% GDP)&         220&       7.564&       3.618&       0.424&       13.17\\
International Tax Revenue (\% GDP)&         220&       0.291&       0.307&      -0.013&       1.087\\
GDP (Current USD, Billions)&         220&        2193&        4505&       53.87&       21540\\
Import Value (Billions USD)&         215&       503.3&       670.2&       16.83&        3121\\
\hline\hline
\end{tabular}
\label{tab:sumstats}
\end{table}
\begin{tablenotes}[flushleft]
\small
\item \textit{Note:} Sample consists of 8 developed countries over 20 years (2001-2020): Australia, France, Israel, Korea, Norway, Romania, Switzerland, and the United States. Tax variables from ICTD Government Revenue Dataset; trade and GDP variables from World Integrated Trade Solution (WITS).
\end{tablenotes}


The World Integrated Trade Solution (WITS) is a trade software provided by the world bank that consists of trade and tariff data. The International Centre for Tax and Development Government Revenue Dataset (ICTD-GRD). Our sample includes eleven developed countries 2001-2020: Australia, Blegium, Canada, France, Ireland, Israel, South Korea, New Zealand, Norway, Switzerland, and the United States, amounting to 220 country-year observations. We classify these countries as "developed" based on their consistently high GDP per capita. 


We obtained four trade indicators to create our key variables from WITS: Exports of goods and services (\% of GDP) measures the value of all exported goods and services as a percentage of GDP; Imports of goods and services (\% of GDP) measures the value of all imported goods and services as a percentage of GDP, GDP (current USD) represents the gross domestic product of in current US dollars; and Import Value (Current USD) captures the total value of imports in current US dollars. 
We also obtained two tax revenue variables from ICTD-GRD: Domestic Tax Revenue (\% of GDP) measures total taxes on goods and services; and International Tax Revenue (\% of GDP) which is taxes on international trade, primarily consisting of tariffs and customs duties on imports. 
\footnote{See the reproducibility package for more detailed instructions 
\protect\href{https://github.com/ecn310/course-project-tariffs/tree/main/Report\#readme}{here.}}

\section{Results}
\label{sec:result}

\setlength{\pdfpagewidth}{8.5in} \setlength{\pdfpageheight}{11in}


\begin{table}[h]
\centering
\caption{Import-to-GDP Ratio by Country (2001-2020)}
\label{tab:import_country}
\begin{tabular}{lrrrr}
\toprule
\textbf{Country} & \textbf{Mean} & \textbf{Std. Dev.} & \textbf{Min} & \textbf{Max} \\
\midrule
Ireland & 84.61 & 15.31 & 65.65 & 111.54 \\
Belgium & 76.64 & 5.76 & 65.34 & 83.49 \\
Switzerland & 53.06 & 5.73 & 42.52 & 61.63 \\
Korea, Rep. & 38.82 & 7.21 & 29.24 & 52.81 \\
Canada & 32.95 & 1.49 & 30.00 & 36.26 \\
\midrule
Israel & 32.59 & 4.80 & 23.37 & 40.11 \\
France & 29.57 & 2.77 & 25.15 & 34.08 \\
Norway & 29.38 & 2.51 & 26.62 & 34.30 \\
New Zealand & 27.76 & 2.14 & 22.59 & 32.75 \\
Australia & 21.76 & 1.19 & 18.84 & 24.07 \\
\midrule
United States & 15.14 & 1.44 & 12.87 & 17.35 \\
\bottomrule
\end{tabular}
\end{table}
\begin{tablenotes}[flushleft]
\small
\item \textit{Note:} Import-to-GDP ratio calculated as (Import Value / GDP Current USD) × 100. Countries sorted by mean import-to-GDP ratio (highest to lowest). Data from World Integrated Trade Solution (WITS), 2001-2020.
\end{tablenotes}


Table \ref{tab:import_country} lists the summary statistics for the variable ImportGDPRatio generated from dividing ImportValue variable observations by RealGDP variable observations, multiplied by 100. Understanding the variation in import dependency across the countries is essential to testing whether the relationship between tariff revenue and domestic tax revenue differs based on a country's trade intensity. The hypothesis predicts that countries more dependent on imports will face greater fiscal pressure when tariff revenues decline, leading to stronger substitution towards toward domestic taxes. The Import-GDP-Ratio measures the extent to which a country's economy relies on imported goods and services. Countries with higher Import-to -GDP ratios should theoretically exhibit stronger negative correlations between tariff and domestic tax revenues. 

Ireland shows the highest import intensity with an average of 84.61 percent over the 20-year period, with Belgium following closely at 76.64 percent and Switzerland had a mean import ratio of 53.06 percent. These three countries represent the most import intensive economies in the sample. 

At the middle tier of import dependency, Korea, Canada, Israel, France, Norway and New Zealand demonstrate moderate import dependency. The lower end of import intensity includes Australia averaging 21.76 and the United States having the lowest import dependency at 15.14 percent. The United states being the largest economy in the sample, demonstrates economic self-sufficiency. 


\begin{table}[h]
\centering
\caption{Correlations Between International Tax and Domestic Tax Revenue by Country}
\label{tab:correlations_individual}
\begin{tabular}{lrrr}
\toprule
\textbf{Country} & \textbf{Correlation} & \textbf{P-value} & \textbf{Import-GDP} \\
\midrule
Korea, Rep. & $0.769$ & 0.0001 & 38.82\% \\
Switzerland & $0.740$ & 0.0002 & 53.06\% \\
Israel & $0.432$ & 0.0574 & 32.59\% \\
Australia & $-0.319$ & 0.1703 & 21.76\% \\
United States & $-0.397$ & 0.0829 & 15.14\% \\
France & $0.268$ & 0.2542 & 29.57\% \\
Belgium & $0.261$ & 0.2662 & 76.64\% \\
Norway & $0.187$ & 0.4302 & 29.38\% \\
New Zealand & $-0.182$ & 0.4427 & 27.76\% \\
Canada & $0.127$ & 0.5945 & 32.95\% \\
Ireland & $0.088$ & 0.7124 & 84.61\% \\
\bottomrule
\end{tabular}
\end{table}
\begin{tablenotes}[flushleft]
\small
\item \textit{Note:} Pearson correlations between International Tax Revenue (\% GDP) and Domestic Tax Revenue (\% GDP) for each country over 2001-2020. *** p$<$0.001. Import-GDP ratios shown for reference.
\end{tablenotes}
% \end{table}
The correlation results between international tax revenue and domestic tax revenue show some interesting patterns. Out of the eleven countries, only two had statistically significant correlations, and they were both positive instead of negative like the hypothesis predicted. Korea had the strongest correlation at 0.769 with a p-value of 0.0001, which means when tariff revenues go up, domestic tax revenues also go up. Switzerland was similar with a correlation of 0.740 and p-value of 0.0002. 


Explain what analyses you did, 
provide evidence (like in the descriptive stats exercise, but refined and clear) and then explain what your results mean.




\section{Discussion}
\label{sec:discussion}

Optional. This is where you would discuss any of the following if they are extensive; otherwise, this material can be put in the Results (caveats) and Conclusions (future work and next steps) sections.
\begin{itemize}
    \item caveats (are there problems with the data that there are no obvious ways to resolve? if so, how might this impact your results?)
    \item future work / next steps
    \item implications of the results: that is, how your findings -- if they were causally identified -- might inform policymaking, etc.
\end{itemize}

\section{Conclusion}
\label{sec:conclusion}

This paper examined how domestic consumption taxes relate to tariff levels in developed countries and whether this depends on a country's GDP and import to GDP ratio.  Using data from the 11 developed countries from 2001-2020, our analysis combined trade indicators from WITS with tax revenue variables from ICTD-GRD.  These countries were ranked according to import intensity to see whether high dependence on imported goods led to higher domestic consumption tax revenue.  

The descriptive statistics show large variation in import to GDP ratios.  The correlation does not provide strong support for our hypothesis.  Only Korea and Switzerland show a strong relationship between international tax revenue and domestic consumption tax revenue, and both correlations were positive instead of negative like we predicted.  For the other countries, the correlations were weak.  These results suggest that even in import dependent economies, domestic taxes and tariffs do not move in opposite directions.  Overall, we found that while theory may predict substitution between tariff revenues and domestic consumption tax revenues, the evidence is mixed and does not strongly support a clear pattern across these developed countries.

\newpage
\section*{Bibliography}
\singlespacing
\setlength\bibsep{0pt}
Michael Keen, Jenny E. Ligthart, Coordinating tariff reduction and domestic tax reform, Journal of International Economics, Volume 56, Issue 2, 2002, Pages 489-507, ISSN 0022-1996
    Guglielmo Maria Caporale, Christophe Rault, Robert Sova, Anamaria Sova, Europe Agreements and Trade Balance: Evidence from Four New EU Members, IZA Discussion Paper No 5683, 2011 (revised 2025), 35 pages.    
 
    Waglé, Swarnim. "Coordinating tax reforms in the poorest countries: Can lost tariffs be recouped?." World Bank Policy Research Working Paper 5919 (2011). 

    Ho, Thuy Tien, Xuan Hang Tran, and Quang Khai Nguyen. ”Tax revenue-economic growth relationship and the role of trade openness in developing countries.” Cogent Business and Management 10(2) (2023)




\newpage
\section*{Data Appendix} \label{sec:appendixa}
\addcontentsline{toc}{section}{Appendix A}

You should at least direct your reader to your replication package. You might put key elements of your replication package in this section as well.

\end{document}