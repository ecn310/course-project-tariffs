\documentclass[12pt]{article}

% set margins and spacing
\addtolength{\textwidth}{1.3in}
\addtolength{\oddsidemargin}{-.65in} %left margin
\addtolength{\evensidemargin}{-.65in}
\setlength{\textheight}{9in}
\setlength{\topmargin}{-.5in}
\setlength{\headheight}{0.0in}
\setlength{\footskip}{.375in}
\renewcommand{\baselinestretch}{1.0}
\linespread{1.0}

% load miscellaneous packages
\usepackage{csquotes}
\usepackage[american]{babel}
\usepackage{booktabs}
\usepackage{threeparttable}
\usepackage[usenames,dvipsnames]{color}
\usepackage{graphicx,amsbsy,amssymb, amsmath, amsthm, MnSymbol,bbding,times, verbatim,bm,pifont,pdfsync,setspace,natbib}

% enable hyperlinks and table of contents
\usepackage[pdftex,
bookmarks=true,
bookmarksnumbered=false,
pdfview=fitH,
bookmarksopen=true,hyperfootnotes=false]{hyperref}

% define environments
\newtheorem{definition}{Definition}
\newtheorem{fact}{Fact}
\newtheorem{result}{Result}
\newtheorem{proposition}{Proposition}



\begin{document}
\title{Insert title here}
\author{Name 1\thanks{Syracuse University, Economics Department. Email: kbuzard@syr.edu.} \and Name 2\thanks{abc} \and Name 3\thanks{abc}}
\date{\vskip-.1in \today}
\maketitle

\vskip.3in
\begin{center} {\bf Abstract} \end{center}

\begin{quote}
{\small Insert abstract text here: 75-200 words, very high-level summary of your project. Look at the three most closely-related papers in your literature review to get an idea of what should be included.}
\end{quote}

\bigskip
\section{Introduction} \label{sec:introduction}

Answer the questions
\begin{enumerate}
    \item \textbf{Why should the reader care? / Why is the topic important?} (required)
    \item \textbf{What question will you answer? How will you do it?} (required)
        \begin{enumerate}
            \item If your theory/hypothesis fit in one paragraph, include it here. If it is longer, make it a separate section after the lit review. EITHER OPTION IS FINE as long as the length is sufficient/appropriate for your project.
        \end{enumerate}
    \item \textbf{What did you find?} (required)
    \item \textbf{Give a "road map" of the paper. Where will the reader find the various parts of your work?} (required)
\end{enumerate}

\section{Literature Review} \label{sec:literature}

Discuss at least seven papers that are closely related to your results (more is better). Explain how they're related. Did you find something similar, or different? Did you look at a different context? Different time period? Different level of detail?
\begin{itemize}
    \item The most important thing to do here is to highlight how what you do compares to these other papers
\end{itemize}

\section{Theoretical Analysis}
\label{sec:theory}
Optional--may include in intro if it's short.


\section{Data}
\label{sec:data}

Consists of Firm-level data from \href{https://wits.worldbank.org/country-indicator.aspx?lang=en}{WITS}
(World Integrated Trade Solution) examined from public reports published by the WITS' TRAINS Tariff Measures and Preference Beneficiaries, UNCTAD, UN Comtrade and the WTO IDB. International Trade Indicators from the WITS were selected to focus on eight Developed countries with relatively high GDP per Capita from 2001-2020: Australia, France, Israel, Romania, South Korea, Norway, Switzerland, and the United States, for a total of one hundred sixty observations.

 The International Trade Indicators interface available on the database was accessed. Upon loading, the Trade Indicators (variables listed above) were accessed individually through the website's integrated "Search For Indicator" search bar. The selected indicators were Exports of goods and services (\% of GDP), Imports of goods and services (\% of GDP), GDP (constant 2010 USD), Taxes on goods and services (current LCU), Taxes on international trade (current LCU), Customs and other import duties (\% of tax revenue) and Imports of Goods and Services (BoP current USD). The Country Timeseries datasets were downloaded as excel files by click the gray download button with an arrow pointing downwards in the top right corner of the online table from the WITS website.

The international trade indicators were downloaded and stored on Github as individual excel files for each correspondingly: Seven excel files were downloaded and saved listing the complete WITS indicator observations recorded for 193 countries across 1988-2022. The seven excel files were each imported into Stata and saved individually as a dta file. All data files were merged using Stata's append command and combined into one dataset.

Reproducibility Do-File available on Github- Click "Access Github Do-File"

Artificial Intelligence was used in the process of making this work.

\href{https://github.com/ecn310/course-project-tariffs/blob/main/Do%20files/Tarrifs%20Timeseries.do}
{Access Github Do-File}

\section{Results}
\label{sec:result}
\begin{table}[h]
\centering
\caption{Summary Statistics of Key Variables (2001-2020)}
\label{tab:summary_base}
\begin{tabular}{lrrrrr}
\toprule
\textbf{Variable} & \textbf{N} & \textbf{Mean} & \textbf{Std. Dev.} & \textbf{Min} & \textbf{Max} \\
\midrule
Tariff Revenue (\% GDP) & 160 & 0.29 & 0.28 & $-0.01$ & 0.98 \\
Domestic Tax Revenue (\% GDP) & 160 & 16.76 & 5.82 & 7.76 & 28.36 \\
International Tax Revenue (\% GDP) & 160 & 0.29 & 0.28 & $-0.01$ & 0.98 \\
Exports (\% GDP) & 160 & 33.68 & 14.76 & 9.04 & 72.07 \\
Imports (\% GDP) & 160 & 32.18 & 11.60 & 13.15 & 61.63 \\
Real GDP (Trillions USD) & 160 & 2.89 & 5.38 & 0.11 & 19.90 \\
Import Value (Billions USD) & 160 & 552 & 766 & 16.2 & 3,120 \\
\bottomrule
\end{tabular}
\begin{tablenotes}[flushleft]
\small
\item \textit{Note:} Sample consists of 8 developed countries over 20 years (2001-2020): Australia, France, Israel, Korea, Norway, Romania, Switzerland, and the United States. Tax variables from ICTD Government Revenue Dataset; trade and GDP variables from World Integrated Trade Solution (WITS).
\end{tablenotes}
\end{table}

\begin{table}[h]
\centering
\caption{Import-to-GDP Ratio by Country (2001-2020)}
\label{tab:import_country}
\begin{threeparttable}
\begin{tabular}{lrrrr}
\toprule
\textbf{Country} & \textbf{Mean} & \textbf{Std. Dev.} & \textbf{Min} & \textbf{Max} \\
\midrule
Switzerland & 47.84 & 13.22 & 24.34 & 65.41 \\
Romania & 38.93 & 12.17 & 14.99 & 52.05 \\
Korea, Rep. & 36.22 & 8.90 & 20.30 & 50.61 \\
France & 31.01 & 6.08 & 17.92 & 37.17 \\
Norway & 30.50 & 7.35 & 15.71 & 39.62 \\
Israel & 29.49 & 3.66 & 23.17 & 35.87 \\
Australia & 18.59 & 5.06 & 8.92 & 26.54 \\
United States & 14.11 & 2.13 & 9.91 & 16.30 \\
\bottomrule
\end{tabular}
\begin{tablenotes}[flushleft]
\small
\item \textit{Note:} Import-to-GDP ratio calculated as (Import Value / Real GDP) × 100. 
Countries sorted by mean import-to-GDP ratio (highest to lowest).  Import Value and Real GDP data from the World Integrated Trade Solution 
\end{tablenotes}
\end{threeparttable}
\end{table}

Table \ref{tab:import_country} lists the summary statistics for the variable ImportGDPRatio generated from dividing ImportValue variable observations by RealGDP variable observations, multiplied by 100. The command "sum ImportGDPRatio if CountryName" was executed for the eight developed countries analyzed. Table 1 ranks countries by mean ImportGDPRatio observed: The four countries listed with the highest mean ImportGDPRatio include Switzerland, Romania, South Korea, and France, while the four countries listed with the lowest mean ImportGDPRatio include Norway, Israel, Australia, and the United States.


\begin{table}[h]
\centering
\caption{Pearson Correlations: Tariff Revenue and Domestic Tax Revenue by Import Intensity}
\label{tab:correlations}
\begin{tabular}{lcc}
\toprule
\textbf{Country Group} & \textbf{Correlation} & \textbf{P-value} \\
\midrule
High Import-GDP & 0.080 & 0.544 \\
\quad (Switzerland, Romania, Korea, France) & & \\
\addlinespace
Low Import-GDP & $-0.125$ & 0.215 \\
\quad (Norway, Israel, Australia, United States) & & \\
\bottomrule\end{tabular}
\begin{tablenotes}[flushleft]
\small
\item \textit{Note:} Neither correlation is statistically significant at the 0.05 level.
\end{tablenotes}
\end{table}




\end{document}
Explain what analyses you did, 
provide evidence (like in the descriptive stats exercise, but refined and clear) and then explain what your results mean.




\section{Discussion}
\label{sec:discussion}

Optional. This is where you would discuss any of the following if they are extensive; otherwise, this material can be put in the Results (caveats) and Conclusions (future work and next steps) sections.
\begin{itemize}
    \item caveats (are there problems with the data that there are no obvious ways to resolve? if so, how might this impact your results?)
    \item future work / next steps
    \item implications of the results: that is, how your findings -- if they were causally identified -- might inform policymaking, etc.
\end{itemize}

\section{Conclusion}
\label{sec:conclusion}

Re-state (in different words) what you did and what you learned. If your discussion (Section 6) would be short, you can just have a Conclusion section that includes your discussion (that is, leave out a separate Discussion section).

\newpage
\section*{Bibliography}
\singlespacing
\setlength\bibsep{0pt}

You can either explicitly include your list of references, or you can learn to use BibTex so that it includes the references automatically.

Either way, this list should include ONLY the papers (reports, book chatpers, etc.) that you actually cite in the text (no extra).

At the same time EVERYTHING you cite in the main text must have an entry here (no references in text that don't have something here).

You can choose which citation style to follow. Whichever you choose, you must follow it consistently, including the convention to always alphabetize by the first author's last name.

\newpage
\section*{Data Appendix} \label{sec:appendixa}
\addcontentsline{toc}{section}{Appendix A}

You should at least direct your reader to your replication package. You might put key elements of your replication package in this section as well.

\end{document}