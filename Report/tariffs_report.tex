\documentclass[12pt]{article}

% set margins and spacing
\addtolength{\textwidth}{1.3in}
\addtolength{\oddsidemargin}{-.65in} %left margin
\addtolength{\evensidemargin}{-.65in}
\setlength{\textheight}{9in}
\setlength{\topmargin}{-.5in}
\setlength{\headheight}{0.0in}
\setlength{\footskip}{.375in}
\renewcommand{\baselinestretch}{1.0}
\linespread{1.0}

% load miscellaneous packages
\usepackage{csquotes}
\usepackage[american]{babel}
\usepackage[usenames,dvipsnames]{color}
\usepackage{graphicx,amsbsy,amssymb, amsmath, amsthm, MnSymbol,bbding,times, verbatim,bm,pifont,pdfsync,setspace,natbib}

% enable hyperlinks and table of contents
\usepackage[pdftex,
bookmarks=true,
bookmarksnumbered=false,
pdfview=fitH,
bookmarksopen=true,hyperfootnotes=false]{hyperref}

% define environments
\newtheorem{definition}{Definition}
\newtheorem{fact}{Fact}
\newtheorem{result}{Result}
\newtheorem{proposition}{Proposition}



\begin{document}
\title{Insert title here}
\author{Name 1\thanks{Syracuse University, Economics Department. Email: kbuzard@syr.edu.} \and Name 2\thanks{abc} \and Name 3\thanks{abc}}
\date{\vskip-.1in \today}
\maketitle

\vskip.3in
\begin{center} {\bf Abstract} \end{center}

\begin{quote}
{\small Insert abstract text here: 75-200 words, very high-level summary of your project. Look at the three most closely-related papers in your literature review to get an idea of what should be included.}
\end{quote}

\bigskip
\section{Introduction} \label{sec:introduction}

Answer the questions
\begin{enumerate}
    \item \textbf{Why should the reader care? / Why is the topic important?} (required)
    \item \textbf{What question will you answer? How will you do it?} (required)
        \begin{enumerate}
            \item If your theory/hypothesis fit in one paragraph, include it here. If it is longer, make it a separate section after the lit review. EITHER OPTION IS FINE as long as the length is sufficient/appropriate for your project.
        \end{enumerate}
    \item \textbf{What did you find?} (required)
    \item \textbf{Give a "road map" of the paper. Where will the reader find the various parts of your work?} (required)
\end{enumerate}

\section{Literature Review} \label{sec:literature}

Discuss at least seven papers that are closely related to your results (more is better). Explain how they're related. Did you find something similar, or different? Did you look at a different context? Different time period? Different level of detail?
\begin{itemize}
    \item The most important thing to do here is to highlight how what you do compares to these other papers
\end{itemize}

\section{Theoretical Analysis}
\label{sec:theory}
Optional--may include in intro if it's short.


\section{Data}
\label{sec:data}

Consists of Country-level data from \href{https://wits.worldbank.org/country-indicator.aspx?lang=en}{WITS}
{(World Integrated Trade Solution)} examined from public reports published by the WITS' TRAINS Tariff Measures and Preference Beneficiaries, UNCTAD, UN Comtrade and the WTO IDB. We selected eight developed countries with relatively high GDP per Capita from 2001-2020 (from the International Trade Indicators from the WITS): Australia, France, Israel, Romania, South Korea, Norway, Switzerland, and the United States, for a total of one hundred sixty observations.

 The International Trade Indicators interface available on the database was accessed. Upon loading, the Trade Indicators (variables listed above) were accessed individually through the website's integrated "Search For Indicator" search bar. The selected indicators were Exports of goods and services (\% of GDP), Imports of goods and services (\% of GDP), GDP (constant 2010 USD), Taxes on goods and services (current LCU), Taxes on international trade (current LCU), Customs and other import duties (\% of tax revenue) and Imports of Goods and Services (BoP current USD). The Country Timeseries datasets were downloaded as excel files by click the gray download button with an arrow pointing downwards in the top right corner of the online table from the WITS website.

We downloaded the international trade indicators and downloaded and stored them on Github as individual excel files for each correspondingly: We downloaded and saved seven excel files listing the complete WITS indicator observations recorded for 193 countries across 1988-2022. We imported the seven excel files into Stata and saved each of them individually as a dta file. Then we merged all data files using Stata's append command and combined into one dataset.

Reproducibility Do-File available on Github- Click "Access Github Do-File"


\href{https://github.com/ecn310/course-project-tariffs/blob/main/Do%20files/Tarrifs%20Timeseries.do}
{Access Github Do-File}

\section{Results}
\label{sec:result}
\begin{table}[htbp]
\centering
\caption{Summary Statistics- Import-to-GDP Ratio by Country}
\label{tab:import_gdp_summary}
\begin{tabular}{lrrrrrr}
Country & Obs & Mean & Std. Dev. & Min & Max \\
Switzerland & 20 & 47.84 & 13.22 & 24.34 & 65.41 \\
Romania & 20 & 38.93 & 12.17 & 14.99 & 52.05 \\
Korea, Rep. & 20 & 36.22 & 8.90 & 20.30 & 50.61 \\
France & 20 & 31.01 & 6.08 & 17.92 & 37.17 \\
Norway & 20 & 30.50 & 7.35 & 15.71 & 39.62 \\
Israel & 20 & 29.49 & 3.66 & 23.17 & 35.87 \\
Australia & 20 & 18.59 & 5.06 & 8.92 & 26.54 \\
United States & 20 & 14.11 & 2.13 & 9.91 & 16.30 \\
\end{tabular}
\begin{tablenotes}
\small
\item \textit{Note:} Import-to-GDP ratio expressed as percentage. Data covers period 2001-2020.

Table 1 demonstrates the substantial variation in import dependency across the eight countries, which is essential for testing our hypothesis that the relationship between tariffs and domestic taxes differs between high and low import dependency countries.
Table 1 lists the summary statistics for the variable ImportGDPRatio generated from dividing ImportValue variable observations by RealGDP variable observations, multiplied by 100. Table 1 ranks countries by mean ImportGDPRatio observed: The four countries listed with the highest mean ImportGDPRatio include Switzerland, Romania, South Korea, and France, while the four countries listed with the lowest mean ImportGDPRatio include Norway, Israel, Australia, and the United States.
\end{tablenotes}
\end{table}
\begin{document}

\begin{{figure}}[h]
    \centering
    \begin{subfigure}{\textwidth}
        \includegraphics[width=0.7\textwidth]{Report/scatterplot_high_importgpdratio.pdf}
        \caption{Tariffs vs Domestic Taxes in High Import-to-GDP Ratio Developed Countries}
Figure \ref{fig:high_import_scatter} includes a scatterplot that examines the variables DomesticTaxP and TariffPTaxRev: WITS international trade indicator data observing the total revenue from domestic taxes as a percent of tax revenue and revenue from tariffs as a percent of total tax revenue. The scatterplot shows a cluster exhibiting a strong negative relationship between the two variables examined. A cluster included the twenty evaluations for the country of South Korea which observed very high Domestic Tax percent very high in comparison to three countries with similar High Import-to-GDP Ratio. The specific cluster listed various Domestic Tax percent observations of over sixty percent and some Tariff Revenue percent observations of over six percent.
    \end{subfigure}
    \hfill
    \begin{subfigure}{\textwidth}
        \includegraphics[width=0.7\textwidth]{Report/scatterplot_high_zoom_importgdpratio.pdf}
        \caption{Tariffs vs Domestic Taxes in High Import-to-GDP Ratio Countries (Zoomed View)}
Figure \ref{fig:high_zoom_import} includes a scatterplot that examines the variables DomesticTaxP and TariffPTaxRev for the three countries with High Import-to-GDP Ratio previously clustered in Figure 1: Switzerland, Romania, and France. The scatterplot shows the three countries exhibiting a negative relationship between the two variables examined albeit weaker.
    \end{subfigure}
\end{figure}
\begin{figure}[h]
    \centering
        \includegraphics[width=\textwidth]{Report/scatterplot_low_importgpdratio.pdf}
        \caption{Tariffs vs Domestic Taxes in Low Import-to-GDP Ratio Developed Countries}
Figure \ref{fig:low_importgdpratio} includes a scatterplot that examines the variables DomesticTaxP and TariffPTaxRev for four countries with Low Import-to-GDP Ratio: Norway, Israel, Australia, and the United States. The scatterplot shows the four countries exhibiting a weak negative relationship between the two variables examined. In comparison with the four High Import-to-GDP Ratio countries observed, the negative relationship appears to be weaker in Low Import-to-GDP Ratio countries.
\end{figure}
\end{document}
Explain what analyses you did, 
provide evidence (like in the descriptive stats exercise, but refined and clear) and then explain what your results mean.




\section{Discussion}
\label{sec:discussion}

Optional. This is where you would discuss any of the following if they are extensive; otherwise, this material can be put in the Results (caveats) and Conclusions (future work and next steps) sections.
\begin{itemize}
    \item caveats (are there problems with the data that there are no obvious ways to resolve? if so, how might this impact your results?)
    \item future work / next steps
    \item implications of the results: that is, how your findings -- if they were causally identified -- might inform policymaking, etc.
\end{itemize}

\section{Conclusion}
\label{sec:conclusion}

Re-state (in different words) what you did and what you learned. If your discussion (Section 6) would be short, you can just have a Conclusion section that includes your discussion (that is, leave out a separate Discussion section).

\newpage
\section*{Bibliography}
\singlespacing
\setlength\bibsep{0pt}

You can either explicitly include your list of references, or you can learn to use BibTex so that it includes the references automatically.

Either way, this list should include ONLY the papers (reports, book chatpers, etc.) that you actually cite in the text (no extra).

At the same time EVERYTHING you cite in the main text must have an entry here (no references in text that don't have something here).

You can choose which citation style to follow. Whichever you choose, you must follow it consistently, including the convention to always alphabetize by the first author's last name.

\newpage
\section*{Data Appendix} \label{sec:appendixa}
\addcontentsline{toc}{section}{Appendix A}

You should at least direct your reader to your replication package. You might put key elements of your replication package in this section as well.

\end{document}