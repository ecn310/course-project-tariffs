\documentclass[12pt]{article}

% set margins and spacing
\addtolength{\textwidth}{1.3in}
\addtolength{\oddsidemargin}{-.65in} %left margin
\addtolength{\evensidemargin}{-.65in}
\setlength{\textheight}{9in}
\setlength{\topmargin}{-.5in}
\setlength{\headheight}{0.0in}
\setlength{\footskip}{.375in}
\renewcommand{\baselinestretch}{1.0}
\linespread{1.0}

% load miscellaneous packages
\usepackage{csquotes}
\usepackage[american]{babel}
\usepackage[usenames,dvipsnames]{color}
\usepackage{graphicx,amsbsy,amssymb, amsmath, amsthm, MnSymbol,bbding,times, verbatim,bm,pifont,pdfsync,setspace,natbib}

% enable hyperlinks and table of contents
\usepackage[pdftex,
bookmarks=true,
bookmarksnumbered=false,
pdfview=fitH,
bookmarksopen=true,hyperfootnotes=false]{hyperref}

% define environments
\newtheorem{definition}{Definition}
\newtheorem{fact}{Fact}
\newtheorem{result}{Result}
\newtheorem{proposition}{Proposition}



\begin{document}
\title{Insert title here}
\author{Name 1\thanks{Syracuse University, Economics Department. Email: kbuzard@syr.edu.} \and Name 2\thanks{abc} \and Name 3\thanks{abc}}
\date{\vskip-.1in \today}
\maketitle

\vskip.3in
\begin{center} {\bf Abstract} \end{center}

\begin{quote}
{\small Insert abstract text here: 75-200 words, very high-level summary of your project. Look at the three most closely-related papers in your literature review to get an idea of what should be included.}
\end{quote}

\bigskip
\section{Introduction} \label{sec:introduction}

Answer the questions
\begin{enumerate}
    \item \textbf{Why should the reader care? / Why is the topic important?} (required)
    \item \textbf{What question will you answer? How will you do it?} (required)
        \begin{enumerate}
            \item If your theory/hypothesis fit in one paragraph, include it here. If it is longer, make it a separate section after the lit review. EITHER OPTION IS FINE as long as the length is sufficient/appropriate for your project.
        \end{enumerate}
    \item \textbf{What did you find?} (required)
    \item \textbf{Give a "road map" of the paper. Where will the reader find the various parts of your work?} (required)
\end{enumerate}

\section{Literature Review} \label{sec:literature}

Discuss at least seven papers that are closely related to your results (more is better). Explain how they're related. Did you find something similar, or different? Did you look at a different context? Different time period? Different level of detail?
\begin{itemize}
    \item The most important thing to do here is to highlight how what you do compares to these other papers
\end{itemize}

\section{Theoretical Analysis}
\label{sec:theory}
Optional--may include in intro if it's short.


\section{Data}
\label{sec:data}

Describe your data. Where you got it from, how it was generated, what variables you'll use, any major steps you had to take (like merging two data sources together).

In a published paper, a lot of this detail will be in a data appendix. For the purposes of this report, you can include it all here unless it's really complicated, in which case you can put the details in your data appendix (this may be the longest section of your report).

At the very least, your data appendix should explain where your processed data, code and documentation is stored; you must reference that appendix from this section so the reader knows where to look at to get more details.

\subsection{Insert high-level reference to your first data set}

Make subsections if your data comes from more than one source and this section becomes longer than three or so paragraphs. If your data is very straightforward and you only have one data source, you may not need any subsections.

\section{Results}
\label{sec:result}

\begin{table}[htbp]
\centering
\caption{Summary Statistics: Import-to-GDP Ratio by Country}
\label{tab:import_gdp_summary}
\begin{tabular}{lrrrrrr}
Country & Obs & Mean & Std. Dev. & Min & Max \\
United States & 20 & 14.11 & 2.13 & 9.91 & 16.30 \\
France & 20 & 31.01 & 6.08 & 17.92 & 37.17 \\
Australia & 20 & 18.59 & 5.06 & 8.92 & 26.54 \\
Switzerland & 20 & 47.84 & 13.22 & 24.34 & 65.41 \\
Norway & 20 & 30.50 & 7.35 & 15.71 & 39.62 \\
Romania & 20 & 38.93 & 12.17 & 14.99 & 52.05 \\
Korea, Rep. & 20 & 36.22 & 8.90 & 20.30 & 50.61 \\
Israel & 20 & 29.49 & 3.66 & 23.17 & 35.87 \\
\end{tabular}
\begin{tablenotes}
\small
\item \textit{Note:} Import-to-GDP ratio expressed as percentage. Data covers period 2001-2020.
\end{tablenotes}
\end{table}


\begin{document}

\begin{figure}[h]
    \centering
    \begin{subfigure}{\textwidth}
        \includegraphics[width=\textwidth]{Overleaf/scatterplot_high_importgpdratio.pdf}
        \caption{Tariffs vs Domestic Taxes in High Import-to-GDP Ratio Developed Countries}
    \end{subfigure}
    \hfill
    \begin{subfigure}{\textwidth}
        \includegraphics[width=\textwidth]{Overleaf/scatterplot_high_zoom_importgdpratio.pdf}
        \caption{Tariffs vs Domestic Taxes in High Import-to-GDP Ratio Countries (Zoomed View)}
    \end{subfigure}
    \label{fig:high_import_comparison}
    \vspace{0.2cm}
    \small
    \textit{Note:} Figure (1) shows the full data range, while Figure (2) provides a zoomed view of the lower range.
\end{figure}
\begin{figure}[h]
    \centering
    \begin{subfigure}{\textwidth}
        \includegraphics[width=\textwidth]{Overleaf/scatterplot_low_importgpdratio.pdf}
        \caption{Tariffs vs Domestic Taxes in Low Import-to-GDP Ratio Developed Countries}
    \end{subfigure}
\end{figure}
\end{document}
Explain what analyses you did, 
provide evidence (like in the descriptive stats exercise, but refined and clear) and then explain what your results mean.




\section{Discussion}
\label{sec:discussion}

Optional. This is where you would discuss any of the following if they are extensive; otherwise, this material can be put in the Results (caveats) and Conclusions (future work and next steps) sections.
\begin{itemize}
    \item caveats (are there problems with the data that there are no obvious ways to resolve? if so, how might this impact your results?)
    \item future work / next steps
    \item implications of the results: that is, how your findings -- if they were causally identified -- might inform policymaking, etc.
\end{itemize}

\section{Conclusion}
\label{sec:conclusion}

Re-state (in different words) what you did and what you learned. If your discussion (Section 6) would be short, you can just have a Conclusion section that includes your discussion (that is, leave out a separate Discussion section).

\newpage
\section*{Bibliography}
\singlespacing
\setlength\bibsep{0pt}

You can either explicitly include your list of references, or you can learn to use BibTex so that it includes the references automatically.

Either way, this list should include ONLY the papers (reports, book chatpers, etc.) that you actually cite in the text (no extra).

At the same time EVERYTHING you cite in the main text must have an entry here (no references in text that don't have something here).

You can choose which citation style to follow. Whichever you choose, you must follow it consistently, including the convention to always alphabetize by the first author's last name.

\newpage
\section*{Data Appendix} \label{sec:appendixa}
\addcontentsline{toc}{section}{Appendix A}

You should at least direct your reader to your replication package. You might put key elements of your replication package in this section as well.

\end{document}